\documentclass[uplatex,a4paper,11pt]{jsarticle}
\usepackage{booktabs}
\usepackage{longtable}
\usepackage{array}
\usepackage[utf8]{inputenc}
\usepackage[T1]{fontenc}
\usepackage{lmodern}
\usepackage[margin=2cm]{geometry}
\usepackage{graphicx}
\usepackage{amsmath}
\usepackage{caption}
\usepackage{listings}
\usepackage{xcolor}  % カラー設定のため
% \usepackage{hyperref}
% \usepackage{url}
% % コードのスタイル設定
\lstset{
  basicstyle=\ttfamily\small, % コードの基本スタイル
  keywordstyle=\color{blue}, % キーワードの色
  commentstyle=\color{gray}, % コメントの色
  stringstyle=\color{red}, % 文字列の色
  numbers=left, % 行番号を左に表示
  numberstyle=\tiny\color{gray}, % 行番号のスタイル
  frame=single, % コードの周りに枠をつける
  breaklines=true, % 行が長い場合に折り返す
  captionpos=b, % キャプションの位置
  tabsize=2, % タブの幅
}

\begin{document}

\begin{titlepage}
    \centering
    \vspace*{1cm}
    
    \Large{\textbf{Title}}\\
    \vspace{1.5cm}
    \Large{Subtitle}\\
	\large{実施日: YY/MM/DD}\\
    \vspace{2cm}
    \large{氏名: Hoge}\\
    \large{学籍番号: 082150521}\\
	\large{hoge hoge}\\
    \vfill
    提出日: \today\\
\end{titlepage}


\end{document}
